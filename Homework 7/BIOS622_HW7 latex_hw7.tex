\documentstyle[epsfig,amsmath,11pt]{article}
%\usepackage{amssymb,psfig,epsfig}
\bibliographystyle{plain}
\setlength{\oddsidemargin}{0.01in}
\setlength{\textwidth}{6.2in}
\setlength{\topmargin}{-0.51in}
\setlength{\textheight}{8.5in}

\begin{document}
\def\bebf{\mbox{\boldmath $\beta$}}
\def\mubf{\mbox{\boldmath $\mu$}}
\def\epbf{\mbox{\scriptsize\boldmath $\epsilon$}}
\def\mus{\mbox{\scriptsize\boldmath $\mu$}}


\title{BIOS 622 Homework 7}
\vspace{0.1in}
\author{Instructor: Sijin Wen\\ }
\date{\today}
\maketitle

\baselineskip=12pt
%\parskip 12pt

\begin{enumerate}



\item The following table gives a small data set of survival times and a covariate $z$:

\begin{center}

\begin{tabular}{|c|c|c|c|}
 \hline
\hline
patient ID & survival time in years & $\delta$ &z
\\
\hline
1	& 7 & 1 & 4\\
\hline
2	&8 &  1&3\\
\hline
3	&9 & 0 & 5\\
\hline
4	&10 & 1& 6\\
\hline
\hline

\end{tabular}
\end{center}

where $\delta = 0$ means a right censored observation. Assuming a Cox proportional hazards model:
$$
\lambda(t|z) = \lambda_0(t) e^{\beta z}
$$

where $\lambda_0(t)$ is baseline hazard function (sometimes, we write it as $h_0(t)$ below in Hint, for example):


\begin{enumerate}
\item Write down the partial likelihood of $\beta$.
\item In R, plot the log partial likelihood of $\beta$ in [-8, 3], and convince yourself that this function is concave.
\item In R, find $\hat{\beta}$ that maximize this log partial likelihood function, calculate the second derivative of
the log partial likelihood function at $\hat{\beta}$.
\item (Optional if you have SAS) Use Phreg in SAS (sample code from `Cox model rationale' in Week 10 folder) to fit the above proportional hazards model to the data. How do your results compare to those from the SAS output?
\item Redo (d) using coxph in R. What is the estimated hazard ratio (i.e., $exp(\hat{\beta}))$ and its 95\%CI?
\item Redo (e) using R code assuming a Weibull model (it is also a proportional hazards model). What is the estimated hazard ratio (i.e., $exp(\hat{\beta}))$ and its 95\%CI? 
\item What is your interpretation about the hazard ratio $exp(\hat{\beta})$? Is the covariate $z$ a continuous variable or a categorical variable?

\end{enumerate}



\end{enumerate}

\parskip 12pt


Hint: here is the example we will discuss in the lecture (on next page):

\begin{center}

\begin{tabular}{|c|c|c|c|}
 \hline
\hline
patient ID & survival time & $\delta$ & z
\\
\hline
1	& 2&1 & 2\\
\hline
2	&2&0 & 2\\
\hline
3	&5&1 & 1\\
\hline
4	& 7&1 & 3\\
\hline
\hline

\end{tabular}
\end{center}

The partial likelihood is
$$
\begin{aligned}
PL(\beta) &= \frac{h_0(2)e^{2\beta}}{h_0(2)e^{2\beta}+h_0(2)e^{2\beta}+h_0(2)e^{\beta} + h_0(2)e^{3\beta}} 
\times  \frac{h_0(5)e^{\beta}}{h_0(5)e^{\beta} +h_0(5)e^{3\beta} } \times \frac{h_0(7)e^{3\beta}}{h_0(7)e^{3\beta}}\\
&=\frac{e^{2\beta}}{e^{2\beta}+e^{2\beta}+e^{\beta} + e^{3\beta}} 
\times  \frac{e^{\beta}}{e^{\beta} +e^{3\beta} } \times \frac{e^{3\beta}}{e^{3\beta}}
\end{aligned}
$$
The key question is `what is $\beta$ to maximize the above function'? Notice that the survival time is not shown in the $PL$ in the final calculation since $h_0(t)e^{\beta z}$ is in the $PL$ but $h_0(t)$ has been canceled out in each components, which implies that we don't need to specify the survival distribution in the Cox model.



\end{document}





