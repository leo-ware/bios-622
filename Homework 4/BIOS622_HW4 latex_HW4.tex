\documentstyle[epsfig,amsmath,11pt]{article}
%\usepackage{amssymb,psfig,epsfig}
\bibliographystyle{plain}
\setlength{\oddsidemargin}{0.01in}
\setlength{\textwidth}{6.5in}
\setlength{\topmargin}{-0.51in}
\setlength{\textheight}{8.5in}

\begin{document}
\def\bebf{\mbox{\boldmath $\beta$}}
\def\mubf{\mbox{\boldmath $\mu$}}
\def\epbf{\mbox{\scriptsize\boldmath $\epsilon$}}
\def\mus{\mbox{\scriptsize\boldmath $\mu$}}


\title{BIOS 622 Homework 4}
\vspace{0.2in}
\author{Instructor: Sijin Wen\\ }
\date{\today}
\maketitle

\baselineskip=18pt
%\parskip 12pt

Please answer each question precisely and completely. It is permissible to discuss each other, but all of your work must be your own.  

\begin{enumerate}


%\item Find p-values in HW3 Q2(c) and Q2(d) using Wald statistic test (or likelihood ratio test if you didn't do it in HW3).


\item When we use Wald statistic to find estimation and its 95\%CI, we assume that sample size is large, thus $\hat{\lambda}$ will approximately have Chi-square or normal distribution (after taking square of root). If the sample size is small, the normal distribution may not be a good approximation. Two alternative approaches are (1) bootstrapping and (2) Bayesian method using MCMC algorithm to get a lot of samples from the ‘true’ distribution of $\lambda$, then we can find $\hat{\lambda}$ using sample-mean, sample-standard-deviation, sample-quantiles for 95\%CI. Also you can visualize the ‘true’ distribution of $\hat{\lambda}$ using histogram on those MCMC samples. A modification of example from the course notes (Page 60). $T_1,...,T_n$ iid exponential($\lambda$). The outcome is time to death from severe viral disease. The modified data are as follows (the same as HW3 Q2):
\begin{itemize}
\item Steroid: 1(2), 1+(2), 4+, 5, 7, 8, 10, 10+, 12+, 16+(3)
\item Control: 1, 2, 3(2), 3+, 5+(2), 16+(8)
\end{itemize}

\begin{enumerate}
\item Find 95\% CI of $\hat{\lambda_i}$ (i = 1, 2) for each treatment group, using bootstrapping approach

\item Find $\hat{\lambda_i}$ (i = 1, 2) and its 95\% CI for each treatment group, using Bayesian approach (you may use the R code below)

\item Based on (a) and (b), estimate hazard ratio between two groups (HR = $\lambda_1/\lambda_2$) 
\item Estimate a 95\% CI of hazard ratio using either bootstrapping or Bayesian approach.
\end{enumerate}

\end{enumerate}

\begin{verbatim}

############## steroid group/data:
t1 <- c(1,1,1,1,4,5,7,8,10,10,12,16,16,16)
cen1 <- c(1,1,0,0,0,1,1,1,1,0,0,0,0,0)
sum(cen1)/sum(t1)  ##MLE of lambda
sum(t1)/sum(cen1)  ##MLE of mu

############## control group:
t2 <- c(1,2,3,3,3,5,5,rep(16,8))
cen2 <- c(1,1,1,1,0,0,0,rep(0,8))   
sum(cen2)/sum(t2)  ## MLE of lambda
sum(t2)/sum(cen2)  ## MLe of mu

##############log-likelihood

loglik <- function(lamda,y,ind){
	res.u <- sum(log(lamda*exp(-lamda*y[ind==1])))
	res.cen <- sum(log(exp(-lamda*y[ind==0])))
	return((res.u+res.cen))
}

###############MCMC 

mcmc.s <- function(y,ind,lam, e0, N)
{
	set.seed(99)
	acc <- 0
	opt <- rep(NA, N)
	for(i in 1:N) {
		print(i)
		old <- loglik(lam,y,ind)
		cand <- lam + e0 * rnorm(1)
		while(cand <0){cand <- lam + e0 * rnorm(1)}
		new <- loglik(cand,y,ind)
		if(log(runif(1)) < new - old) {
			lam <- cand
			acc <- acc + 1
		}		
		opt[i] <- lam
}
	par(mfrow = c(2, 1))
	plot(1:N, opt, type = "l", xlab = "", ylab = "lambda")
	plot(1:N, 1/opt, type = "l", xlab = "", ylab = "mu")
	print(acc/N)
	return(opt)  ##lambda
}

junk<-mcmc.s(y=t1,ind=cen1,lam=sum(cen1)/sum(t1) ,e0=0.09,10000)
mean(junk[-(1:5000)])  ##lambda
sd(junk[-(1:5000)])   ## sd of lambda
sum(cen1)/sum(t1)  ## comparing MLE 

junk2<-mcmc.s(y=t2,ind=cen2,lam=sum(cen2)/sum(t2),e0=0.03,10000)
mean(junk2[-(1:5000)])  ## lambda2
sd(junk2[-(1:5000)])  ## sd
sum(cen2)/sum(t2)  ## comparing MLE 

hist(junk[-(1:5000)]/junk2[-(1:5000)])  ## histogram of hazrd ratio?


\end{verbatim}



\end{document}





